\documentclass[12pt,a4paper]{article}
\parskip 4.2pt
\parindent 8.4pt  % Sets leading space for paragraphs.
\usepackage[font=sf]{caption} % Changes font of captions.
\usepackage{amsmath}
\usepackage{amsfonts}
\usepackage{amssymb}
\usepackage{siunitx}
\usepackage{verbatim}
\usepackage{hyperref} % Required for inserting clickable links.
\usepackage{natbib} % Required for APA-style citations.
\usepackage[top=2cm, left=1.5cm, right=1.5cm, bottom=2cm]{geometry}
\usepackage[backend=biber]{biblatex}
\addbibresource{example.bib} 
\usepackage{hyperref}

\usepackage[T2A]{fontenc}
\usepackage[utf8]{inputenc}
\usepackage[english,russian]{babel}

\title{Предсказание цены криптовалюты с помощью временных рядов}


\author{Работу выполнили:\\Югансон Наталья, Макарова Ксения\\ НИС БЭАД221}
\date{11 ноября 2023 г.}
\begin{document}
\maketitle


\newpage
\tableofcontents

\newpage

\section{Введение}
Прогнозирование цены криптовалют представляет значительный интерес для многих людей. Это можно делать с помощью временных рядов, что мы и покажем в этой работе. Проанализируем данные за определенный период времени, затем по этим данным построим предсказание и узнаем, насколько наше предположение оказалось верным.

\section{Задача}
 Предсказать цену на криптовалюту используя разные методы и узнать, какой больше всего подходит для нашего случая. Из датасета известно, в какой момент времени была какая цена. После обработки этих данных мы прогнозируем цену на будущее. Следующим шагом мы считаем среднюю абсолютную ошибку(mae) и узнаем корректность нашего результата. Сравнивая ошибки, понимаем, какой метод привел к лучшему результату.
 
\section{Описание и подготовка данных}
Датасет содержит данные о цене криптовалюты в течение 547 дней и насколько цена менялась за этот день. Все данные разделяем на тренировочную и тестовую часть. На тренировочной мы будем считать, что должно получиться, а на тестовой проверять, насколько хорошо мы предсказали. Также нужно сформировать данные о предыдущем значении ряда (лаговые признаки) и среднем значение ряда по дню (агрегированные признаки).

\section{Исследуемые методы}
\subsection{Статистические методы}
Все статистические методы подходят для стационарных рядов. Поэтому первый шаг это  привести ряд к стационарному, затем с помощью линейной регрессии строим прогноз для стационарного ряда. И применяем обратные преобразования к прогнозу. Для нашего датасета mae получается приблизительно 800. Попробуем еще!

\subsection{Методы машинного обучения}
Существует много методов машинного обучения для прогнозирования, мы использовали бустинг. Для прогноза в частности использовали библиотеку catboost, модель CatBoostRegressor. Mae получилось 227. Первый график - это спрогнозированная цена на последние 7 дней, а на втором показано, насколько отличается наше предсказанная цена от той, которая будет на самом деле.

\section{Результаты}
Нет универсального метода, который всегда считал бы цену криптовалюты с минимальной ошибкой, но для нашего случая метод с градиентным бустингом, а именно его частным случаем - моделью catboost оказался лучшим. Этот метод мощен тем, что в процессе работы построил композицию из слабых моделей, деревьев решений. Он способен выявлять сложные нелинейные зависимости в данных, что невозможно сделать вручную. Думаем, текущий результат можно было бы улучшить, если учитывать контекст данной задачи - политический (войны, конфликты), социальные волнения, период анализа, выбранный горизонт прогнозирования и в целом особенности данных. Конечно, на будущее стоило попробовать комбинацию нескольких методов или тщательнее настроить параметры моделей. 

\section{Источники}

\subsection{links on literature:}

1) \href{https://ieeexplore.ieee.org/abstract/document/8404760}{Seçkin Karasu, Aytaç Altan. \emph{Prediction of Bitcoin prices with machine learning methods using time series data}. IEEE, 2018.} (дата обращения: 08.11.2023)

2) \href{https://onlinelibrary.wiley.com/doi/abs/10.1002/isaf.1488}{Murali Mallikarjuna Rao Perumalla, Sanjay Kumar Singh, Aditya Khamparian. \emph{Cryptocurrency price prediction using traditional statistical and machine-learning techniques}. The Journal of Finance, 2021.} (дата обращения: 09.11.2023)


3)\href{https://www.sciencedirect.com/science/article/abs/pii/S0045790620307576}{Ahmed Ibrahim, Rasha Kashef. \emph{Predicting market movement direction for bitcoin: A comparison of time series modeling methods}. The Journal of Computers \& Electrical Engineering, 2021.} (дата обращения: 09.11.2023)

    

4)\href{https://www.sciencedirect.com/science/article/abs/pii/S0045790620307576}{Betul AYGUN. \emph{Comparison of Statistical and Machine Learning Algorithms for Forecasting Daily Bitcoin Returns}. The European Journal of Science and Technology, 2021.} (дата обращения: 10.11.2023)

\subsection{online courses used:}

\href{http://wiki.cs.hse.ru/Time_series_modelling_22_23}{\emph{some lectures from fcs course (3d year)}}


\end{document}